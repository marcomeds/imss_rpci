 \documentclass[oneside,11pt]{article}

\input{preamble.tex}
%%% HELPER CODE FOR DEALING WITH EXTERNAL REFERENCES
\usepackage{xr}
\makeatletter
\newcommand*{\addFileDependency}[1]{
  \typeout{(#1)}
  \@addtofilelist{#1}
  \IfFileExists{#1}{}{\typeout{No file #1.}}
}
\makeatother


\newcommand*{\myexternaldocument}[1]{
    \externaldocument{#1}
    \addFileDependency{#1.tex}
    \addFileDependency{#1.aux}
}

%\myexternaldocument{OA}

%%%%%%%%%%%%%%%%%%%%%%%%%%%%%%%% DOCUMENT
\begin{document}

\title{IMSS RPCI \thanks{We want to thank.}}
\author{Eduardo Alcaraz \and Gabriela López \and Luis Martínez \and Marco Medina \and Enrique Seira  \thanks{Seira: ITAM, enrique.seira@gmail.com (corresponding author); Alcaraz: IMSS, eduardo.alcarazp@imss.gob.mx; López: IMSS, ; Martínez: IMSS, luis.martinezch@imss.gob.mx; Medina:  ITAM, marco.medina@itam.mx}}
\date{This draft:  \today \\[2 cm]}

%\vspace{.5in}


\maketitle
\thispagestyle{empty}
\begin{abstract}

%Abstract here. 

\end{abstract}

\vspace{.3in}

\textbf{Keywords: }

\textbf{JEL codes:}

\newpage

\pagenumbering{arabic}
\etocdepthtag.toc{mtchapter}
\etocsettagdepth{mtchapter}{subsection}
\etocsettagdepth{mtappendix}{none}

\section{Introduction} \label{introduction}


\clearpage

\section{Conclusion} \label{conclusion}

%%%%%%%%%%%%%%%%%%%%%%%%%%%%%%%%%%%%%%%%%%%%%%%%%%%%%%%%%%%%%

\newpage

%%%%%%%%%%%%%%%%%%%%%%%%%%%%%%%%%%%%%%%%%%%%%%%%%%%%%%%%%%%%%
%BIBLIOGRAPHY

\clearpage
\bibliographystyle{ecta}
%\bibliographystyle{authordate1}
%\bibliographystyle{amsalpha}
%\bibliographystyle{AEA}

\bibliography{References}

%\FloatBarrier
%%%%%%%%%%%%%%%%%%%%%%%%%%%%%%%%%%%%%%%%

\clearpage
\singlespacing

\section{Tables}

\subsection{Summary Statistics}

\begin{table}[H]
\footnotesize
\centering
\begin{threeparttable}
\centering
\caption{Summary Statistics\label{tab:summary_stats_rpci}}
%\textit{Do file: summary_stats_rpci.do}

\begin{tabular}[t]{@{}l}
\toprule
\toprule
\begin{tabular}[t]{lccc}
\input 03_Tables/muestra_10porciento/summary_stats_rpci
\midrule
Workers & & & 1,412,210\\
Firms & & & 339,884\\
\end{tabular}

\tabularnewline 
\bottomrule
\bottomrule

\end{tabular}

\begin{tablenotes}
\setlength\labelsep{0pt}
\scriptsize
\item \textit{Notes}: This table shows summary statistics on selected variables for our final sample. \textit{Sample:} Panel data for a random sample of the workers enrolled at the Mexican Institute of Social Security (IMSS) during 2020 and January 2021 (before the RPCI launch). \textit{Registered for RPCI} is a dummy where 1 means worker $i$ registered for the RPCI at some point in our sample. \textit{Enrolled} is a dummy variable where 1 means worker $i$ was enrolled at IMSS during period $t$. \textit{Women}, \textit{Outsourcing} and \textit{Eventual} are dummies where 1 means worker $i$ is a woman, an outsourced worker or an eventual worker, respectively. \textit{Wage} is registered wage for worker $i$ during period $t$. \textit{N} is the number of non-missing observations for each variable. Wage and worker characteristics are only available if the worker was registered during period $t$. \textit{Workers} and \textit{Firms} are the number of unique workers and firms in our sample. This table is referenced in %\hyperref[subsec:workers]{Section} \ref{subsec:workers}.
\end{tablenotes}
\end{threeparttable}
\end{table}

\clearpage

\subsection{RPCI effect on enrollment and wages}

\begin{table}[H]
\footnotesize
\centering
\begin{threeparttable}
\centering
\caption{RPCI effect on enrollment and wages\label{tab:dcdh_rpci}}
%\textit{Do file: event_study_rpci.do}

\begin{tabularx}{0.75\textwidth}[t]{@{}l@{}l@{}l}
\toprule
\toprule
\begin{tabular}[t]{p{0.2\textwidth}P{0.15\textwidth}}
& Enrolled \\
\midrule
\input 03_Tables/muestra_10porciento/dcdh_alta
\end{tabular}
&
\begin{tabular}[t]{HP{0.15\textwidth}}
& Formal Wage \\
\midrule
\input 03_Tables/muestra_10porciento/dcdh_sal_formal
\end{tabular}
&
\begin{tabular}[t]{HP{0.15\textwidth}}
& Wage$^\dagger$ \\
\midrule
\input 03_Tables/muestra_10porciento/dcdh_sal_cierre
\end{tabular}

\tabularnewline 
\bottomrule
\bottomrule

\end{tabularx}

\begin{tablenotes}
\setlength\labelsep{0pt}
\scriptsize
\item \textit{Notes}: This table shows the effect of registering to the RPCI on enrollment and the worker's wage. \textit{Sample:} Panel data for a random sample of the workers enrolled at the Mexican Institute of Social Security (IMSS) during during 2020 and January 2021 (before the RPCI launch). \textit{Enrolled} is a dummy variable where 1 means worker $i$ was enrolled at IMSS during period $t$. $\dagger$ \textit{Formal Wage} and \textit{Wage} are the registered wage for worker $i$ during period $t$, the difference is \textit{Formal Wage} is 0 when the worker isn't enrolled, while \textit{Wage} is missing when the worker isn't enrolled. The coefficient displayed is the average treatment effect estimated following \cite{de2020two}, using the robust dynamic option to account for possible heterogeneous treatment effects across cohorts. The number of observations are the number of differences of the outcome and of the treatment used in the estimation. The effect is the average effect of the treatment across the switchers. Switchers is the number of switchers this effect applies to. Robust standard errors clustered by worker id in parenthesis. *** $p<0.01$, ** $p<0.05$, * $p<0.1$. This table is referenced in %\hyperref[subsec:workers]{Section} \ref{subsec:workers}.
\end{tablenotes}
\end{threeparttable}
\end{table}

\clearpage

\subsection{RPCI effect on wage changes}

\begin{table}[H]
\footnotesize
\centering
\begin{threeparttable}
\centering
\caption{RPCI effect on wage changes\label{tab:dcdh_wage_changes_rpci}}
%\textit{Do file: yearly_volatility_rpci.do}

\begin{tabularx}{0.75\textwidth}[t]{@{}l@{}l@{}l}
\toprule
\toprule
\begin{tabular}[t]{p{0.2\textwidth}P{0.15\textwidth}}
& Wage Changes \\
\midrule
\input 03_Tables/muestra_10porciento/dcdh_sal_diff_yr
\end{tabular}
&
\begin{tabular}[t]{HP{0.15\textwidth}}
& Wage Raises \\
\midrule
\input 03_Tables/muestra_10porciento/dcdh_sal_mayor_yr
\end{tabular}
&
\begin{tabular}[t]{HP{0.15\textwidth}}
& Wage Cuts \\
\midrule
\input 03_Tables/muestra_10porciento/dcdh_sal_menor_yr
\end{tabular}

\tabularnewline 
\bottomrule
\bottomrule

\end{tabularx}

\begin{tablenotes}
\setlength\labelsep{0pt}
\scriptsize
\item \textit{Notes}: This table shows the effect of registering to the RPCI on wage changes per year. \textit{Sample:} Panel data for a random sample of the workers enrolled at the Mexican Institute of Social Security (IMSS) during 2020 and January 2021 (before the RPCI launch). \textit{Wage Changes} counts the number of changes in the reported wage at IMSS of worker $i$ during year $t$. \textit{Wage Raises} counts the number of wage changes where the wage increased and \textit{Wage Cuts} counts the number of wage changes where the wage decreased. The coefficient displayed is the treatment effect on the year after being treated estimated following \cite{de2020two}. The number of observations are the number of differences of the outcome and of the treatment used in the estimation. The effect is the average effect of the treatment across the switchers. Switchers is the number of switchers this effect applies to. Robust standard errors clustered by worker id in parenthesis. *** $p<0.01$, ** $p<0.05$, * $p<0.1$. This table is referenced in %\hyperref[subsec:wage_changes]{Section} \ref{subsec:wage_changes}.
\end{tablenotes}
\end{threeparttable}
\end{table}

\clearpage


\singlespacing

\section{Figures}

\subsection{Knowledge about IMSS}

\vspace{.7in}
\begin{figure}[H]
    \centering
    \caption{Knowledge about IMSS and worker's reported wages \label{fig:hist_knowledge_register_survey}}
    \includegraphics[width=\textwidth]{04_Figures/worker_survey/hist_knowledge_register_survey.pdf}
\end{figure}
\scriptsize{\textit{Notes}: This figure shows answers to questions about IMSS and wage reporting from the worker survey. \textit{Sample:} 233,709 answers from a survey conducted via email to workers enrolled at IMSS during August 2021. Questions 1-2, about the worker's employer, included the option "I don't know". Questions 3-5 ask about the worker's actions or knowledge didn't include the option "I don't know".}

\clearpage

\subsection{RPCI}

\begin{figure}[H]
    \caption{RPCI example}
    \label{rpci_example}
    \begin{center}
    
    \begin{subfigure}{0.49\textwidth}
    \caption{RPCI within the IMSS Digital app}
    \includegraphics[width=\textwidth]{04_Figures/rpci_app/rpci_2.png}
    \end{subfigure}
    \begin{subfigure}{0.49\textwidth}
    \caption{RPCI PDF file}
    \includegraphics[width=\textwidth]{04_Figures/rpci_app/rpci_3.png}
    \end{subfigure}
    

    \end{center}
\end{figure}
\scriptsize{
\noindent Figure (a) shows the IMSS Digital app, where once the worker is registered for the RPCI, the worker can download their report in PDF or receive it via email. Figure (b) shows an example of the PDF for the RPCI. The report includes the worker job registered information, such as wage and the firm the worker is registered at.
}

\clearpage

\subsection{RPCI registers by month}

\begin{figure}[H]
    \caption{RPCI registers by month}
    \label{hist_download}
    \begin{center}
    \includegraphics[width=0.75\textwidth]{04_Figures/muestra_1porciento/hist_download_month.pdf}
    \end{center}
\end{figure}
\scriptsize{
\noindent This figure shows the total number of workers registered for the RPCI. The right y-axis measures the fraction of workers who registered for the RPCI during each month from the total workers who registered for the RPCI. The left y-axis measures the cumulative number of workers who registered for the RPCI.
}

\clearpage

\subsection{Event Studies: RPCI effect on enrollment and wages}

\begin{figure}[H]
    \centering
    \caption{Event studies - RPCI effect on enrollment and wages \label{fig:event_study_rpci}}
    
    \begin{subfigure}{0.49\textwidth}
    \caption{Effect on being enrolled}
    \includegraphics[width=\textwidth]{04_Figures/muestra_10porciento/event_study_alta_dcdh.pdf}
    \end{subfigure}
    
    \begin{subfigure}{0.49\textwidth}
    \caption{Effect on formal wage}
    \includegraphics[width=\textwidth]{04_Figures/muestra_10porciento/event_study_sal_formal_dcdh.pdf}
    \end{subfigure}
    \begin{subfigure}{0.49\textwidth}
    \caption{Effect on wage$^\dagger$}
    \includegraphics[width=\textwidth]{04_Figures/muestra_10porciento/event_study_sal_cierre_dcdh.pdf}
    \end{subfigure}
    
    %\textit{Do file: event_study_rpci.do}
\end{figure}

\scriptsize{
\noindent \textit{Notes}: This figures shows the event studies for the effect of registering to the RPCI on enrollment and the worker's wage. \textit{Sample:} Panel data for a random sample of the workers enrolled at the Mexican Institute of Social Security (IMSS) during during 2020 and January 2021 (before the RPCI launch). \textit{Enrolled} is a dummy variable where 1 means worker $i$ was enrolled at IMSS during period $t$. $\dagger$ \textit{Formal Wage} and \textit{Wage} are the registered wage for worker $i$ during period $t$, the difference is \textit{Formal Wage} is 0 when the worker isn't enrolled, while \textit{Wage} is missing when the worker isn't enrolled. The event studies follow the estimators proposed by \cite{de2020two}, using the robust dynamic option to account for possible heterogeneous treatment effects across cohorts. Robust standard errors clustered by worker id. This figure is referenced in %\hyperref[subsec:workers]{Section} \ref{subsec:workers}.
}

\clearpage

\subsection{Event Studies: RPCI effect on wage changes}

\begin{figure}[H]
    \centering
    \caption{Event studies - RPCI effect on wage changes \label{fig:event_study_wage_changes_rpci}}
    
    \begin{subfigure}{0.49\textwidth}
    \caption{Effect on wage changes}
    \includegraphics[width=\textwidth]{04_Figures/muestra_10porciento/event_study_sal_diff_yr_dcdh.pdf}
    \end{subfigure}
    
    \begin{subfigure}{0.49\textwidth}
    \caption{Effect on wage raises}
    \includegraphics[width=\textwidth]{04_Figures/muestra_10porciento/event_study_sal_mayor_yr_dcdh.pdf}
    \end{subfigure}
    \begin{subfigure}{0.49\textwidth}
    \caption{Effect on wage cuts}
    \includegraphics[width=\textwidth]{04_Figures/muestra_10porciento/event_study_sal_menor_yr_dcdh.pdf}
    \end{subfigure}
    
    %\textit{Do file: event_study_rpci.do}
\end{figure}

\scriptsize{
\noindent \textit{Notes}: This figures shows the event studies for the effect of registering to the RPCI on enrollment and the worker's wage. \textit{Sample:} Panel data for a random sample of the workers enrolled at the Mexican Institute of Social Security (IMSS) during during 2020 and January 2021 (before the RPCI launch). \textit{Wage Changes} counts the number of changes in the reported wage at IMSS of worker $i$ during year $t$. \textit{Wage Raises} counts the number of wage changes where the wage increased and \textit{Wage Cuts} counts the number of wage changes where the wage decreased. The event studies follow the estimators proposed by \cite{de2020two}, using the robust dynamic option to account for possible heterogeneous treatment effects across cohorts. Robust standard errors clustered by worker id. This figure is referenced in %\hyperref[subsec:workers]{Section} \ref{subsec:workers}.
}

\clearpage

\subsection{Heterogeneity by worker characteristics}

\begin{figure}[H]
    \centering
    \caption{Heterogeneity by worker characteristics \label{fig:heterogeneity_worker_rpci}}
    
    \begin{subfigure}{\textwidth}
    \includegraphics[width=\textwidth]{04_Figures/muestra_10porciento/dcdh_heterogeneity_worker_characteristics.pdf}
    \end{subfigure}
    
    %\textit{Do file: dcdh_heterogeneity_rpci.do}
\end{figure}

\scriptsize{
\noindent \textit{Notes}: This figure explores heterogeneity in the effect of registering to the RPCI on enrollment and the worker's wage by baseline worker's characteristics (characteristics during 2020, before the RPCI launch). \textit{Sample:} Panel data for a random sample of the workers enrolled at the Mexican Institute of Social Security (IMSS) during during 2020 and January 2021 (before the RPCI launch). \textit{Wage Changes} counts the number of changes in the reported wage at IMSS of worker $i$ during year $t$. \textit{Wage Raises} counts the number of wage changes where the wage increased and \textit{Wage Cuts} counts the number of wage changes where the wage decreased. The coefficient displayed is the treatment effect on the year after being treated estimated following \cite{de2020two}. The number of observations are the number of differences of the outcome and of the treatment used in the estimation. The effect is the average effect of the treatment across the switchers. Switchers is the number of switchers this effect applies to. Robust standard errors clustered by worker id. *** $p<0.01$, ** $p<0.05$, * $p<0.1$. This figure is referenced in %\hyperref[subsec:workers]{Section} \ref{subsec:workers}.
}

\clearpage

\subsection{Heterogeneity by firm characteristics}

\begin{figure}[H]
    \centering
    \caption{Heterogeneity by firm characteristics \label{fig:heterogeneity_firm_rpci}}
    
    \begin{subfigure}{\textwidth}
    \includegraphics[width=\textwidth]{04_Figures/muestra_10porciento/dcdh_heterogeneity_firm_characteristics.pdf}
    \end{subfigure}
    
    %\textit{Do file: dcdh_heterogeneity_rpci.do}
\end{figure}

\scriptsize{
\noindent \textit{Notes}: This figure explores heterogeneity in the effect of registering to the RPCI on enrollment and the worker's wage by baseline firm's characteristics (characteristics during 2020, before the RPCI launch). \textit{Sample:} Panel data for a random sample of the workers enrolled at the Mexican Institute of Social Security (IMSS) during during 2020 and January 2021 (before the RPCI launch). \textit{Wage Changes} counts the number of changes in the reported wage at IMSS of worker $i$ during year $t$. \textit{Wage Raises} counts the number of wage changes where the wage increased and \textit{Wage Cuts} counts the number of wage changes where the wage decreased. The coefficient displayed is the treatment effect on the year after being treated estimated following \cite{de2020two}. The number of observations are the number of differences of the outcome and of the treatment used in the estimation. The effect is the average effect of the treatment across the switchers. Switchers is the number of switchers this effect applies to. Robust standard errors clustered by worker id. *** $p<0.01$, ** $p<0.05$, * $p<0.1$. This figure is referenced in %\hyperref[subsec:workers]{Section} \ref{subsec:workers}.
}

\clearpage

%%%%%%%%%%%%%%%%%%%%%%%%%%%%%%%%%%%%%%%%%%%%%%%%%%%%%%

\newpage
%  \documentclass[oneside,11pt]{article}

% \input{preamble.tex}
% %%% HELPER CODE FOR DEALING WITH EXTERNAL REFERENCES
% \usepackage{xr}
% \makeatletter
% \newcommand*{\addFileDependency}[1]{
%   \typeout{(#1)}
%   \@addtofilelist{#1}
%   \IfFileExists{#1}{}{\typeout{No file #1.}}
% }
% \makeatother


% \newcommand*{\myexternaldocument}[1]{
%     \externaldocument{#1}
%     \addFileDependency{#1.tex}
%     \addFileDependency{#1.aux}
% }

% %\myexternaldocument{OA}

% %%%%%%%%%%%%%%%%%%%%%%%%%%%%%%%% DOCUMENT
% \begin{document}

%%%%%%%%%%%%%%%%%%%%%%%%%%%%%%%%%%%%%%%%%%%%%%%

% APPENDIX 
\setcounter{table}{0}
\setcounter{figure}{0}
\setcounter{section}{0}
\pagenumbering{gobble}


\begin{center}
	\LARGE IMSS RPCI \\[0.5em]
	\Large{Appendix $-$ For Online Publication} \\[1em]
	\large \author{Eduardo Alcaraz \and Gabriela López \and Luis Martínez \and Marco Medina \and Enrique Seira}
\end{center}

\appendix
\pagenumbering{arabic}
\renewcommand\thefigure{OA-\arabic{figure}}
\renewcommand\thetable{OA-\arabic{table}}
\renewcommand*{\thepage}{OA - \arabic{page}}
\renewcommand\thesection{Appendix \Alph{section}.}
\renewcommand\thesubsection{\Alph{section}.\arabic{subsection}}

%\renewcommand{\cftparskip}{0em} % NOT NEEDED
\renewcommand\cftsecdotsep{\cftdotsep}
\renewcommand\cftsubsecdotsep{\cftnodots}
\renewcommand{\cftsecnumwidth}{6em}
 \renewcommand{\cftpnumalign}{r}
%\renewcommand{\cftsecleader}{\normalfont\cftdotfill{\cftsecdotsep}}


\renewcommand{\cftsecleader}{\cftdotfill{\cftsecdotsep}\hspace{1.8em}}
%\renewcommand{\cftsecpagefont}{20em}
%\renewcommand{\cftfignumwidth}{6em}
%\renewcommand{\cfttabnumwidth}{3.3em}

%\tableofcontents
\etocdepthtag.toc{mtappendix}
\etocsettagdepth{mtchapter}{none}
\etocsettagdepth{mtappendix}{subsection}

\setstretch{0.9}
%\renewcommand\contentsname{} % the empty name

\begingroup
\let\clearpage\relax
%\vspace{-1.5em} % the removed space. Set as appropriate
\tableofcontents
\endgroup

\clearpage

\section{ RPCI}
\vspace{.2in}

\begin{figure}[H]
    \caption{RPCI flyers}
    \label{rpci_flyers}
    \begin{center}
    
    \begin{subfigure}{0.49\textwidth}
    \caption{RPCI flyer titled "Does my employer has me registered at IMSS?"}
    \includegraphics[width=\textwidth]{04_Figures/rpci_app/rpci_flyer_3.jpeg}
    \end{subfigure}
    \begin{subfigure}{0.49\textwidth}
    \caption{RPCI flyer titled "Digital services for a healthy environment"}
    \includegraphics[width=\textwidth]{04_Figures/rpci_app/rpci_flyer_2.jpeg}
    \end{subfigure}

    \end{center}
\end{figure}
\scriptsize{
\noindent Flyers circulated by the Mexican Institute of Social Security (IMSS) for the RPCI. Both flyers explain how you can track and access your job register information, such as wage and firm you are registered at, if you register for the RPCI.
}

\clearpage

\begin{figure}[H]
    \caption{Registering for the RPCI}
    \label{rpci_register}
    \begin{center}
    
    \begin{subfigure}{0.9\textwidth}
    \includegraphics[width=\textwidth]{04_Figures/rpci_app/rpci_register.png}
    \end{subfigure}

    \end{center}
\end{figure}
\scriptsize{
\noindent Diagram shows how to register for the RPCI within the IMSS Digital app. The worker registers only once to access the RPCI, using his Unique Population Registry Key (CURP) and email address.
}

\clearpage

\begin{figure}[H]
    \caption{RPCI example}
    \label{rpci_example}
    \begin{center}
    
    \begin{subfigure}{0.49\textwidth}
    \caption{RPCI within the IMSS Digital app}
    \includegraphics[width=\textwidth]{04_Figures/rpci_app/rpci_2.png}
    \end{subfigure}
    \begin{subfigure}{0.49\textwidth}
    \caption{RPCI PDF file}
    \includegraphics[width=\textwidth]{04_Figures/rpci_app/rpci_3.png}
    \end{subfigure}
    

    \end{center}
\end{figure}
\scriptsize{
\noindent Figure (a) shows the IMSS Digital app, where once the worker is registered for the RPCI, the worker can download their report in PDF or receive it via email. Figure (b) shows an example of the PDF for the RPCI. The report includes the worker job registered information, such as wage and the firm the worker is registered at.
}



\section{Heterogeneity by worker characteristics}

\begin{figure}[H]
    \centering
    \caption{Heterogeneity by worker characteristics \label{fig:heterogeneity_worker_rpci}}
    
    \begin{subfigure}{\textwidth}
    \includegraphics[width=\textwidth]{04_Figures/muestra_10porciento/dcdh_heterogeneity_worker_characteristics_paper.pdf}
    \end{subfigure}
    
    %\textit{Do file: dcdh_heterogeneity_rpci.do}
\end{figure}

\scriptsize{
\noindent \textit{Notes}: This figure explores heterogeneity in the effect of registering to the RPCI on enrollment and the worker's wage by baseline worker's characteristics (characteristics during 2020, before the RPCI launch). \textit{Sample:} Panel data for a random sample of the workers enrolled at the Mexican Institute of Social Security (IMSS) during during 2020 and January 2021 (before the RPCI launch). \textit{Enrolled} is a dummy variable where 1 means worker $i$ was enrolled at IMSS during period $t$. $\dagger$ \textit{Formal Wage} and \textit{Wage} are the registered wage for worker $i$ during period $t$, the difference is \textit{Formal Wage} is 0 when the worker isn't enrolled, while \textit{Wage} is missing when the worker isn't enrolled. The coefficient displayed is the average treatment effect estimated following \cite{de2020two}, using the robust dynamic option to account for possible heterogeneous treatment effects across cohorts. 95\% confidence intervals shown. Robust standard errors clustered by worker id. *** $p<0.01$, ** $p<0.05$, * $p<0.1$. This figure is referenced in %\hyperref[subsec:workers]{Section} \ref{subsec:workers}.
}

\clearpage

\section{Heterogeneity by firm characteristics}

\begin{figure}[H]
    \centering
    \caption{Heterogeneity by firm characteristics \label{fig:heterogeneity_firm_rpci}}
    
    \begin{subfigure}{\textwidth}
    \includegraphics[width=\textwidth]{04_Figures/muestra_10porciento/dcdh_heterogeneity_firm_characteristics_paper.pdf}
    \end{subfigure}
    
    %\textit{Do file: dcdh_heterogeneity_rpci.do}
\end{figure}

\scriptsize{
\noindent \textit{Notes}: This figure explores heterogeneity in the effect of registering to the RPCI on enrollment and the worker's wage by baseline firm's characteristics (characteristics during 2020, before the RPCI launch). \textit{Sample:} Panel data for a random sample of the workers enrolled at the Mexican Institute of Social Security (IMSS) during during 2020 and January 2021 (before the RPCI launch). \textit{Enrolled} is a dummy variable where 1 means worker $i$ was enrolled at IMSS during period $t$. $\dagger$ \textit{Formal Wage} and \textit{Wage} are the registered wage for worker $i$ during period $t$, the difference is \textit{Formal Wage} is 0 when the worker isn't enrolled, while \textit{Wage} is missing when the worker isn't enrolled. The coefficient displayed is the average treatment effect estimated following \cite{de2020two}, using the robust dynamic option to account for possible heterogeneous treatment effects across cohorts. 95\% confidence intervals shown. Robust standard errors clustered by worker id. *** $p<0.01$, ** $p<0.05$, * $p<0.1$. This figure is referenced in %\hyperref[subsec:workers]{Section} \ref{subsec:workers}.
}


%\clearpage

%\bibliographystyle{authordate1}
%\bibliographystyle{amsalpha}
%\bibliographystyle{AER}

%\bibliography{References}




% \end{document}

\end{document}

